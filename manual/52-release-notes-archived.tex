\newpage
%%%%%%%%%%%%%%%%%%%%%%%%%%%%%%%%%%%%
\section{r2021.002}
Released on September 10, 2021, this version addresses compatibility with system software 2.9.1.236, adds CLI options for transferring documents, and improves PDF handling and rendering.

\subsection{Compatibility}
\begin{tabular}{ r | l }
  Hardware & reMarkable 1 \& 2 \\
  Software & 1.8.1.1--2.9.1.236 \\
  PC & FreeBSD 13.0, Ubuntu 18/20 LTS, Fedora 33, openSUSE 15.2, CentOS 7, \\
  & macOS 10.13, Windows 10 \\
\end{tabular}

%% \subsection{Important Notices}
%% \begin{itemize}
%% \item[!]{Sample notice.}
%% \end{itemize}

\subsection{Release Notes}
\begin{itemize}
\item{Compatibility with system software 2.9.1.236, rm2fb.\footnote{This is a shim for other third-party software (not RCU) to display graphics on RM2. Read more: \textit{\href{https://github.com/ddvk/remarkable2-framebuffer}{ddvk/remarkable2-framebuffer}}.}}
  \item{Snap-highlighted text extracts into PDF highlight comments.}
  \item{High Density export option (4x DPI bitmap rendering).}
  \item{Create folders and sub-folders.}
\item{Sticky favorited documents and collections on-top.}
\item{Removed pop-up notice when connection becomes interrupted.}
  \item{Check for pseudo-deleted files and prompt non-cloud users to purge.}
    \item{New CLI options: \textit{\--\--cli}, \textit{\--\--screenshot-p}, \textit{\--\--screenshot-l}, \textit{\--\--list-documents}, \textit{\--\--list-collections}, \linebreak\textit{\--\--download-doc}, \textit{\--\--export-pdf-b}, \textit{\--\--export-pdf-v}, \textit{\--\--export-pdf-o}, \textit{\--\--upload-doc}, \linebreak\textit{\--\--upload-doc-to}, \textit{\--\--render-rmn-pdf-b}, \textit{\--\--render-rmn-pdf-v}}
    \item{Show error message for About Pane network failures.}
\item{Improved PDF handling with regards to CropBox, MediaBox.}
\item{Improved handling with reMarkable Cloud.}
\item{Bundle SSL certificates.}
  \item{Sanitize quote characters from filenames.}
\item{Fixed defect where Mechanical Pencil v1 (from system software prior to 1.8.1.1) rendered incorrectly.}
  \item{Fixed defect where RM2 screenshots were mis-aligned by 8 pixels.}
  \item{Fixed defect where using Adjust View had no effect on exported PDFs.}
  \item{Fixed defect where the recovery OS was dependent upon contents of eMMC. Tablet now shows black screen during recovery operations.}
  \item{Fixed defect where connection presets could disappear.}
  \item{Fixed defect where RMN uploads were invisible if they were originally taken from a parent folder.}
    \item{Fixed defect where the program could crash for macOS 11 (Big Sur) users when exporting select PDFs.}
    

\end{itemize}

\subsection{Known Defects}
\begin{itemize}
\item{Some documents exported with a base PDF and Layers enabled may render OK, but be reported in Adobe Acrobat as having a generic error, and the Background layer may toggle the entire page.}
\item{Improper annotation rotation with PDF pages that are rotated 270 degrees (usually appearing upside-down).}
\item{Some PDFs are unable to be decoded and fail to export, or export blank. Some examples are password-protected PDFs, and My Deep Guide's yearly planner.\footnote{A temporary workaround is to process these PDFs with \textit{qpdf} and overwrite the original PDF on the device through SSH.}}
  \item{Lucida Console is forced on macOS, instead of the default system typeface, because of a font-rendering issue in Qt with macOS 11 (Big Sur) where space characters that appear after a comma or period are invisible.}
  \item{No warning for RMPKG fault}
\end{itemize}






\newpage
%%%%%%%%%%%%%%%%%%%%%%%%%%%%%%%%%%%%
\section{r2021.001}
\label{sec:r2021.001}
Released on January 31, 2021, this version addresses compatibility for reMarkable 2, system software 2.5.0.27, and macOS Big Sur, fixes bugs and annoying behavior, and introduces new features.

\subsection{Compatibility}
\begin{tabular}{ r | l }
  Hardware & reMarkable 1 \& 2 \\
  Software & 1.8.1.1--2.5.0.27 \\
  PC & FreeBSD 12.1, Ubuntu 18/20 LTS, Fedora 33, openSUSE 15.2, CentOS 7, \\
  & macOS 10.13-11.1, Windows 10 \\
\end{tabular}

\subsection{Important Notices}
\begin{itemize}
\item[!]{Low-level backup/restore is unsupported with reMarkable 2.}
\item[!]{reMarkable 2 owners on Windows do not need to install the libusb driver.}
\end{itemize}

\subsection{Release Notes}
\begin{itemize}
\item{reMarkable 2 compatibility, screenshots}
  \item{System software 2.5.0.27 compatibility}
  \item{macOS 11 (Big Sur) compatibility}
  \item{KDE, dark theme support}
    \item{Parabola-rM backup/restore}
  \item{Directly import SVG and PNG images as templates}
  \item{Multiple presets for different networks}
  \item{Password visibility toggle}
  \item{Hierarchical download/export documents}
  \item{CLI flags: \textit{\--\--autoconnect}, \textit{\--\--dark}, \textit{\--\--no-compat-check}}
  \item{Rename documents and collections}
  \item{Application icon, GNU/Linux installers}
  \item{Fetch new compatibility tables without updating RCU}
  \item{Switches program license to GNU AGPLv3 or later}
  \item{Uses to .app packaging for macOS}
  \item{Makefile conveniences, like venv and packaging}
  \item{Keyboard shortcuts for quitting/closing}
  \item{Fixes defect where key-based authentication was used instead of password}
  \item{Fixes defect where data-only restore was non-functional}
  \item{Fixes defect where \textit{/home/root} was sometimes used instead of \textit{\$HOME}}
  \item{Fixes defect where backups under Windows failed when running RCU from a non-primary volume}
  \item{Fixes defect where upload progress meter never changed from 0\%}
\end{itemize}

\subsection{Known Defects}
\begin{itemize}
\item{Unable to load recovery OS when the tablet has a botched splash image.\footnote{A new recovery OS is available by email, but was not polished enough to make this release.}}
\item{Some documents exported with a base PDF and Layers enabled may render OK, but be reported in Adobe Acrobat as having a generic error, and the Background layer may toggle the entire page.}
\item{Force-refreshing the Notebooks Pane doesn't work under FreeBSD or macOS.}
\item{No conflict check for RMPKGs}
\item{No warning for RMPKG fault}
\end{itemize}




\newpage
%%%%%%%%%%%%%%%%%%%%%%%%%%%%%%%%%%%%
\section{r2020.003}
\label{sec:r2020-003}
Released on October 3, 2020, this version addresses compatibility for system software 2.3.0.16, fixes bugs and annoying behavior, and introduces new export options.

\subsection{Compatibility}
\footnote{This version is likely compatible with reMarkable 2 (except backup/restore) but is untested.}
\begin{tabular}{ r | l }
  Hardware & RM100, RM102 (reMarkable 1) \\
  Software & 1.8.1.1--2.3.0.16 \\
  PC & FreeBSD 12.1, Ubuntu 18.04, macOS 10.13, Windows 10 \\
\end{tabular}

\subsection{Important Notices}
\begin{itemize}
\item[!]{This version changes the settings and data paths. To migrate old backups, copy the \textit{backups} directory from the old data directory to the new one. The old data directories are:}
  \begin{itemize}
  \item[]{FreeBSD, GNU/Linux: \textit{\textasciitilde/.config/rcu}}
  \item[]{macOS: \textit{\textasciitilde/Library/Preferences}}
  \item[]{Windows: \textit{C:\textbackslash HKEY\_CURRENT\_USER\textbackslash SOFTWARE\textbackslash rcu\textbackslash rcu}}
  \end{itemize}
\end{itemize}

\subsection{Release Notes}
\begin{itemize}
\item{System software 2.3.0.16 compatibility}
\item{Resizable UI, with Hi-DPI support}
\item{New application data directory, user-setable in configuration file}
  \begin{itemize}
  \item{FreeBSD, GNU/Linux}
    \begin{itemize}
    \item[]{Settings: \textit{\textasciitilde/.config/davisr/rcu.conf}}
    \item[]{Data: \textit{\textasciitilde/.local/share/davisr/rcu}}
    \end{itemize}
  \item{macOS}
    \begin{itemize}
    \item[]{Settings: \textit{\textasciitilde/Library/Preferences/rcu.plist}}
    \item[]{Data: \textit{\textasciitilde/Library/Application Support/rcu}}
    \end{itemize}
  \item{Windows}
    \begin{itemize}
    \item[]{Settings: \textit{HKEY\_CURRENT\_USER\textbackslash SOFTWARE\textbackslash davisr\textbackslash rcu}}
    \item[]{Data: \textit{\%APPDATA\%\textbackslash davisr\textbackslash rcu}}
    \end{itemize}
  \end{itemize}
\item{Faster startup/load time}
\item{Automatic notebook view refresh (force with Ctrl+R or F5)}
\item{Drag-and-drop document organization within the Notebooks Pane}
\item{Save last-used directory for import/export operations}
\item{Save notebook sorting method}
\item{Upload to any collection (document folder)}
\item{No longer requires Host to be an IP address in the Connection Dialog}
\item{A port may be given in the Host field, separated with a ':' (colon)}
\item{New Export PDF options}
  \begin{itemize}
  \item{PDF annotations for highlights}
  \item{PDF layers (OCG) for each document layer}
  \item{Customizable ink colors}
  \item{Auto-open exported PDF}
  \end{itemize}
\item{Fix Wallpaper Pane bugs}
  \begin{itemize}
  \item{Removed "Powered Off" overlay text}
  \item{Set to Original button works as-expected and will always keep a backup}
  \end{itemize}
\item{Fix rM Cloud compatibility bugs (delete, re-uploading old archives)}
\item{Sanitize export filenames}
\item{Compatibility with more GNU/Linux distros}
  \begin{itemize}
  \item{Now requires libxcb as a dependency (install \textit{libxcb-xinerama0})}
  \end{itemize}
\item{Warning popup when the tablet is disconnected}
\item{reMy, from which RCU gets its .lines parser, is now credited visibly in the About Pane}
\item{Reduce logging verbosity}
\item{Smoother lines when exporting vectors}
\end{itemize}

\subsection{Known Defects}
\begin{itemize}
\item{Some documents exported with a base PDF and Layers enabled may report in Adobe Acrobat as having a generic error, but otherwise render OK.}
\item{Force-refreshing the Notebooks Pane doesn't work under macOS.}
\item{No conflict check for RMPKGs}
\item{No warning for RMPKG fault}
\end{itemize}



%%%%%%%%%%%%%%%%%%%%%%%%%%%%%%%%%%%%
\newpage
\section{r2020.002}
\label{sec:r2020-002}
Released on September 6, 2020, this version addresses issues blocking users from running \nameref{sec:r2020-001}.

\subsection{Compatibility}
\begin{tabular}{ r | l }
  Hardware & RM100, RM102 (reMarkable 1) \\
  Software & 1.8.1.1--2.2.0.48 \\
  PC & FreeBSD 12.1, Ubuntu 18.04, macOS 10.13, Windows 10 \\
\end{tabular}

\subsection{Release Notes}
\begin{itemize}
\item{Fix "unknown document version" bug}
\item{Fix improper handling of tall PDF documents}
\item{Warn user when not taking/restoring backup from USB}
\item{Set executable permission in macOS release archive}
\item{Update manual with recovery mode instructions}
\item{Update manual with Windows driver instructions}
\end{itemize}

\subsection{Known Defects}
\begin{itemize}
\item{No conflict check for RMPKGs}
\item{No warning for RMPKG fault}
\end{itemize}









%%%%%%%%%%%%%%%%%%%%%%%%%%%%%%%%%%%%
\newpage
\section{r2020.001}
\label{sec:r2020-001}
Released on September 5, 2020, this is the first released version of RCU.

\subsection{Compatibility}
\begin{tabular}{ r | l }
  Hardware & RM100, RM102 (reMarkable 1) \\
  Software & 1.8.1.1--2.2.0.48 \\
  PC & FreeBSD 12.1, Ubuntu 18.04, macOS 10.13, Windows 10 \\
\end{tabular}

\subsection{Release Notes}
\begin{itemize}
\item{First release of reMarkable Connection Utility (RCU)}
\item{Application will open with console window}
\end{itemize}

