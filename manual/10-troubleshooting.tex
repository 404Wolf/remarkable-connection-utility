\chapter{Troubleshooting}
\label{sec:troubleshooting}

\subsection{RCU can't connect to the tablet.}
If a connection between the PC and reMarkable can't be established, the Connection Dialog will show an error message.

The most-common problem is trying to use incorrect login information. This information is found on the tablet, in \textit{Menu}--\textit{Settings}--\textit{Help}--\textit{Copyrights and licenses}, under the section titled \textit{GPLv3 Compliance}. If using a USB connection, the Host will be \textit{10.11.99.1}, the Username will be \textit{root}, and the Password will vary depending on the tablet. The font used to show this password can be confusing if certain characters are used, such as uppercase ``I'', lowercase ``L'', and number ``1''. Try different combinations if your password contains these characters.

If you are certian the login information is correct, it is recommended to use \textit{ping} to see if your PC is able to communicate with the tablet. Another way to test network connectivity is to enable the USB web interface, under \textit{Menu}--\textit{Settings}--\textit{Storage}. If you can connect using the web interface, you likely don't have a network problem.

If you use a PC supplied by an employer, they may be blocking network requests. In that case, it is best to speak with the IT department about granting access.


\subsection{RCU is showing a Compatibility Warning screen.}
When a reMarkable tablet receives an update to the latest system software, RCU may display a Compatibility Warning screen. Since RCU trails reMarkable updates, it defaults to block access if using the program with a reMarkable system software version it doesn't know about. This mechanism is to prevent RCU from clobbering the tablet if reMarkable had introduced a breaking update.

If a tablet update is found to be non-breaking, then RCU does not need to be wholly updated. Instead, RCU can download a new compatibility table that will allow it to work with new tablet software. This table may be updated in the \nameref{sec:aboutpane} with the Fetch Compatibility button.

This warning message may be bypassed with the \textit{\--\--no-compat-check} command-line flag.

\subsection{Templates aren't included with exported notebooks.}
If a compatibility warning is bypassed for the \nameref{sec:notebookspane}, but not for the \nameref{sec:templatespane}, then templates will not be included in PDF or RMN exports. Both panes must be loaded for templates to become exported.

\subsection{macOS won't run RCU because it's from an unidentified developer.}
\label{sec:macosgatekeeper}
Most macOS users have a program called Gatekeeper enabled. This software prevents unsigned-by-Apple software from running. In order to run RCU for the first time on macOS, users typically need to \textit{Right-Click}--\textit{Open}--\textit{Open}, which will allow the user to run the program. After doing that once, a user may simply double-click on the RCU icon thereafter.

The author of RCU, instead of using Apple-specific technology, includes a platform-agnostic PGP signature to verify authenticity. If RCU was downloaded using HTTPS from \textit{files.davisr.me}, then it is reasonably safe to run.


\subsection{Documents created or modified with system software 3.0+ can't be exported as Bitmap or Vector PDFs.}
Support for exporting PDFs through RCU's custom renderer (Bitmap and Vector export options) from documents created or modified with system software 3.0+ is experimental. The document types which may be rendered with RCU's internal renderer are Epubs and PDFs with a reMarkable-native aspect ratio, and notebooks which do not use the continuous pages or text input features.

Any document may be exported with the reMarkable's own renderer by using the Export PDF (Web UI) option. Please refer to \nameref{sec:notebookspane} for information about export types.


\subsection{The ``Export PDF (Web UI)'' option is disabled.}
RCU may export documents through the tablet's web interface (UI) either over USB or Wi-Fi. However, the interface's state may only be changed when the tablet is plugged into a computer with USB.

To enable the tablet's web UI, connect the tablet to a computer with USB. Then, under \textit{Menu}--\textit{Settings}--\textit{Storage}, enable \textit{USB Web Interface}.
