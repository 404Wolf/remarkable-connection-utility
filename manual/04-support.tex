\chapter{Support Information}
This manual, and the RCU software, are available online from the \href{http://www.davisr.me/projects/rcu/}{official project page}.

\section{General Support}

Customers of RCU's original author are entitled to some email support. The author will try his best to satisfy each customer. Please write an email using the following header fields and reference the transaction ID in the message body.

\vspace{0.5cm}
\begin{tabular}{rl}
To:& Davis Remmel \textless d@visr.me\textgreater \\
Subject:& RCU Support
\end{tabular}


\section{Getting Updates}

Updates for release versions are announced via email to eligible customers via email. Additionally, notices are posted to the project's \href{http://www.davisr.me/projects/rcu/}{web page} (RSS available). The program will not update automatically, but new versions may be identified from the \nameref{sec:aboutpane}.

Customers will receive any published updates for a minimum of 365 days from the purchase date. The RCU program will never stop working; updates provide improvements, but a user can never be locked out of the software they already own.


\section{Defect Reporting}

For users who can identify a fault with the program, and its specific circumstance of action, please submit a defect report via email with the following header fields. In the message body, please include: (a) a description of what is seen when using the program, (b) what is expected to be seen when using the program, (c) steps to reproduce the problem, and (d) information about the operating system and RCU version number (found in the \nameref{sec:aboutpane}).

\vspace{0.5cm}
\begin{tabular}{rl}
To:& Davis Remmel \textless d@visr.me\textgreater \\
Subject:& RCU Defect: \textit{Short description of problem}
\end{tabular}


\section{Mailing Lists}
There are two mailing lists: RCU-Announce, and RCU-Develop. All customers are automatically subscribed to the RCU-Announce list.

RCU-Announce only broadcasts when there are new release (stable) versions of the program. Old versions of RCU are linked in this archive.

Subscribing to RCU-Develop is optional. It is a bidirectional list meant to solicit feedback on beta features, discuss general program development, and speed up the release cycle.

\vspace{0.5cm}
\begin{tabular}{rl}
RCU-Announce& \href{https://lists.davisr.me/mailman/listinfo/rcu-announce}{https://lists.davisr.me/mailman/listinfo/rcu-announce} \\
RCU-Develop& \href{https://lists.davisr.me/mailman/listinfo/rcu-develop}{https://lists.davisr.me/mailman/listinfo/rcu-develop}
\end{tabular}
\vspace{0.5cm}

To subscribe to RCU-Develop, please write an email with the following header fields and reference the transaction ID in the message body.

\vspace{0.5cm}
\begin{tabular}{rl}
To:& Davis Remmel \textless d@visr.me\textgreater \\
Subject:& Subscribe to RCU-Develop
\end{tabular}


\section{Contributing Patches}
Since RCU is free software, users may create and share modifications. Some users may wish for their modifications with others, or to have them be re-incorporated into the official RCU software.

The source code of RCU is canonically distributed as a tarball, but is also available as a Git repository. The easiest way to make contributions is to submit diffs to the RCU-Develop mailing list.

\begin{enumerate}
\item{Starting with the RCU development version source code, make changes and log all commits with relevant details.

Commit messages should contain a concise title and descriptive body, explaining not only a summary of the changes, but also why the changes were made. Each commit should be digestible and generally not change more than 250 lines.}

\item{Send an email to the RCU-Develop mailing list using \textit{\href{https://git-scm.com/docs/git-send-email}{git-send-email}}.}

\end{enumerate}

The author watches this mailing list for all submissions. Others are encouraged to create a discussion around submissions. There is no guarantee that submissions will be accepted. Smaller patches have a higher chance of being accepted.
