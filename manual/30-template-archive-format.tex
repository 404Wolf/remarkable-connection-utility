\chapter{Template Archive Format}
\label{sec:templatearchiveformat}
The Template Archive (RMT) format stores a vector image and reMarkable UI metadata in a single file. Typically, reMarkable templates are shared as singular PNG images; this has a number of drawbacks, like being stuck with a static resolution and lack of metadata.

The RMT format solves those issues by bundling a vector image of the template (SVG), instead of a bitmap image, and includes template metadata (JSON) in a tape-archive (TAR). An example file structure of an RMT file may be seen in Figure \ref{fig:rmtstructure}.

\begin{figure}[h]
  \dirtree{%
    .1 ExampleTemplate.rmt.
    .2 template.json.
    .2 template.svg.
}
\caption{Example structure of a template archive}
\label{fig:rmtstructure}
\end{figure}

The \textit{template.json} file should contain, at minimum, a structure similar to Figure \ref{fig:exampletemplatejson}. The \textit{fileName} attribute should be a UUID and is necessary to prevent template collisions; if this is not explicitly specified, one will be generated by RCU.\footnote{Do not share manually-created RMT files that don't have a \textit{fileName} attribute, otherwise collisions may occur or there could be multiple versions of identical templates.} The \textit{categories} array may contain any of the following strings:

\begin{itemize}
\item{Creative}
\item{Grids}
\item{Life/organize}
\item{Lines}
\end{itemize}


\begin{figure}[h]
\begin{minted}[
  mathescape,
  linenos,
  numbersep=5pt,
  gobble=2,
  frame=lines,
  framesep=2mm,
  fontsize=\footnotesize]{json}
  {
      "categories": [
          "Creative"
      ],
      "iconCode": "\ue9d5",
      "landscape": false,
      "name": "Example Template",
      "fileName": "<UUID>"
  }
\end{minted}
\caption{Example source code for \textit{template.json}}
\label{fig:exampletemplatejson}
\end{figure}

The \textit{template.svg} file should contain valid SVG data and have a viewport resolution of 1404$\times$1872 pixels. When a template archive is uploaded, these files are not extracted to the default template location (\textit{/usr/share/remarkable/templates}). Instead, they are extracted into the user's home directory at \textit{\textasciitilde/.local/share/remarkable/templates}. Upon upload, RCU will convert the SVG image to a PNG image to retain compatibility with Xochitl.

If the tablet receives an update that clears the system template directory, the templates will become unavailable from the interface. RCU may detect this condition and prompt the user to recreate these template links, in-effect restoring the templates (they do not need to be re-uploaded).
