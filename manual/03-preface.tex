\newpage

\thispagestyle{empty}\mbox{}\newpage

\renewcommand{\thechapter}{\roman{chapter}}
\chapter{Preface}

You may have seen my work in the reMarkable tablet hacking community. I’m the one who \href{http://www.davisr.me/projects/remarkable-microsd/}{added a microSD card to the tablet}, then later installed a \href{http://www.davisr.me/projects/parabola-rm/}{desktop GNU/Linux environment}. My efforts focus on making the device usable in a general computing context.

This software, reMarkable Connection Utility (RCU), unshackles its users from\break reMarkable's proprietary cloud. The benefits are numerous: snapshots are low-level and full, personal notes never leave the owner's control, users may personalize their device, and totally new features like sharing notebooks are possible. These are things the manufacturer won't provide.

Unlike restrictive black-box software, RCU gives its users \textit{freedom}. Under the terms of its license, the \nameref{sec:license}, users hold the freedom to use it for any purpose, read the source code, modify it, share it with whomever they wish, or even re-sell it---as long as they pass forward these same freedoms. This viral licensing forms a web of non-restrictive (\textit{free}) software, leading the world toward transparency and trust, precipitating \textit{rights} for software users.

If you are a privacy-minded individual who wants to support independent free software development, please \href{http://www.davisr.me/projects/rcu/}{buy RCU}. The funds generated will support me through writing a non-proprietary handwriting recognition engine, eventually authoring ``magic paper'' software influenced by Dynabook.

I would greatly appreciate your purchase; thank you.

\vspace{1.5cm}

\rightline{Davis Remmel}
\rightline{Author of RCU}
